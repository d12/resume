\documentclass[12pt]{extarticle}
\usepackage[margin=0.4in]{geometry}
\usepackage{multicol}
\usepackage{hyperref}
\usepackage{paralist}
\setlength{\columnseprule}{0.4pt}
\pagenumbering{gobble}
\begin{document}
{\fontfamily{lmss}\selectfont
    \begin{center}
        {\LARGE {NATHANIEL WOODTHORPE}}
        \\
        \vspace{0.52cm}
        \hrule
        \vspace{0.4cm}

        \setlength{\tabcolsep}{12pt}
        \begin{tabular}{c c c}
            \href{https://github.com/d12}{https://github.com/d12} & \href{https://www.linkedin.com/in/nwoodthorpe/}{https://www.linkedin.com/in/nwoodthorpe/} & njwoodthorpe@gmail.com
        \end{tabular}
        \vspace{0.22cm}
        \\
        \hrule
    \end{center}
    \vspace{0.5cm}
    {\large \textbf{Experience}}\\
    
    \vspace{-0.2cm}
    {\indent
        \textbf{BricksVR} \hfill \textbf{Feb 2021 - Sept 2021}

        Founder \& Lead Developer

        \begin{compactitem}
            \setlength{\itemindent}{0.5cm}
            \item[--] Managed a team of 7 to build and launch a multiplayer social experience for Oculus VR headsets.
            \item[--] Developed the front-end in Unity and C\#, the API in Ruby, and the infrastructure on GCP.
            \item[--] Built and maintained an active community around the product with over 1,000 users.
            \item[--] Launched the product in June 2021 to several thousand paying users worldwide.
        \end{compactitem}
    }

    \vspace{0.3cm}
    {\indent
        \textbf{Shopify} \hfill \textbf{May 2020 - Feb 2021}

        Senior Software Developer

        \begin{compactitem}
            \setlength{\itemindent}{0.5cm}
            \item[--] Led a team of backend Rails engineers to launch Shopify Capital in Great Britain. 
            \item[--] Worked closely with design, product, and legal to keep alignment and set sprint goals.
            \item[--] Re-architected the Shopify Capital legal agreement system from the ground-up to ensure compliance.
        \end{compactitem}
    }

    \vspace{0.3cm}
    {\indent
        \textbf{GitHub} \hfill \textbf{May 2019 - May 2020}

        Program Manager

        \begin{compactitem}
            \setlength{\itemindent}{0.5cm}
            \item[--] Led a team of 6 developing GitHub Classroom, an edutech tool used by over 100,000 users monthly.
            \item[--] Collaborated with engineering, design, and leadership to set and work towards OKRs.
            \item[--] Wrote project briefs, blog posts and email newsletter campaigns.
            \item[--] Mentored and onboarded new engineers and engineering interns.
        \end{compactitem}
    }

    \vspace{0.3cm}

    {\indent
        \textbf{GitHub} \hfill \textbf{May 2017 - May 2019}

        Software Engineer

        \begin{compactitem}
            \setlength{\itemindent}{0.5cm}
            \item[--] Developed new GraphQL/REST APIs and improved existing ones, focusing on performance and usability.
            \item[--] Worked on the core "repositories" team doing feature requests, performance improvements and bug fixes.
            \item[--] Gave technical and non-technical interviews for engineer and engineering manager candidates.
            \item[--] Wrote several blog posts and gave technical talks at engineering all-hands.
            \item[--] Mentored 6 engineering interns. Offered 1-1's, pairing sessions, and technical mentorship.
        \end{compactitem}
    }

    \vspace{1cm}
    \noindent {\large \textbf{Projects}}\\

    \vspace{-0.2cm}
    {\indent
        \textbf{SMB1 Genetic Learning AI} \hfill \textbf{Feb 2020}

        \begin{compactitem}
            \setlength{\itemindent}{0.5cm}
            \item[--] Built an AI that teaches itself how to play SMB1 from scratch as an exercise to learn about AI.
            \item[--] The AI is a feed-forward neural network (FFNN) trained by a genetic algorithm built in Ruby \& Java.
        \end{compactitem}
    }

     \vspace{0.3cm}
     
    {\indent
        \textbf{GraphQL Remote Loader, an Alternative to Schema Stitching} \hfill \textbf{July 2017}

        \begin{compactitem}
            \setlength{\itemindent}{0.5cm}
            \item[--] Performant, batched GraphQL queries from within the resolvers of a graphql-ruby API.
            \item[--] Uses the graphql-batch gem to batch all requested data into a single outbound GraphQL query.
            \item[--] Presented the project at a GraphQL meetup in Toronto.
            \item[--] The gem is currently in-use at multiple large companies, including Shopify.
        \end{compactitem}
    }

    \vspace{0.3cm}
    \noindent
    {\indent
        \href{https://github.com/d12}{\textbf{Find more projects on my GitHub}}
    }

    \vspace{0.8cm}
    \noindent {\large \textbf{Education}}\\

    \vspace{-0.2cm}
    {\indent
        \textbf{University of Waterloo} Bachelor of Computer Science Co-op (Incomplete) \hfill \textbf{2015-2017}
    }
}
\end{document}
